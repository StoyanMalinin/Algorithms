\documentclass[12pt]{article}
\usepackage{lingmacros}
\usepackage{tree-dvips}
\usepackage[utf8]{inputenc}
\usepackage[russian]{babel}
\usepackage{amsmath,amssymb}
\usepackage{multirow}
\usepackage{hyperref}
\usepackage{caption}
\usepackage{tabularx}

\begin{document}

\paragraph*{}
Текущият анализ е по-подробно обяснение на сложността на авторските решения, предложени за задача sequences на контролното за С група през 2022 година.

\section*{Част 1}
\paragraph*{}
Трябва да покажем, че сред числата $[\frac{n}{1}], [\frac{n}{2}], ..., [\frac{n}{n}]$ са не повече от $2 \sqrt{n}$. Ще покажем това като покажем, че всяко число от вида $[\frac{n}{x}]$ се съдържа в множеството $N_1 = \{1, 2, ..., [\sqrt{n}] \}$ или $N_2 = \{ [\frac{n}{1}], [\frac{n}{2}], ..., [\frac{n}{\sqrt{n}}] \}$.

\subsection*{Случай 1: $x \geq \sqrt{n}$}
\paragraph*{}
Това е лесно. Ясно е, че $\frac{n}{x} \leq \frac{n}{\sqrt{n}} = \sqrt{n}$. Тоест е ясно, че $[\frac{n}{x}] \in N_1$

\subsection*{Случай 2: $x < \sqrt{n}$}
В този случай $\frac{n}{x} > \sqrt{n}$. В този случай $[\frac{n}{x}] \in N_2$ буквално по дефиниция.

\paragraph*{}
Като интуиция за това доказателство можем да ползваме, че всъщност деленето реално е нещо като симетрия спрямо корен квадратен от числителя. И всички начини да разменим $n$ на по-малко положително цяло число могат да се разделят на два случая - когато числителят е малък(малко на брой случаи) или когато числителят е голям(много случаи, но резултатът от деленето е малко число).

\section*{Част 2}
\paragraph*{}
Трябва да покажем, че числата от вида $[\frac{x}{i}]$, където $x \in N_1 \cup N_2$ и $1 \leq i \leq x$, са около $n^{\frac{3}{4}}$. 
Отново ще цепим на два случая.

\subsection*{Случай 1: $x \in N_1$}
\paragraph*{}
Тук директно можем да ползваме Част 1. Знаем, че $x \leq \sqrt{n}$. Знаем, че има най много $2 \sqrt{x} \leq 2 \sqrt{\sqrt{n}}$ числа от вида $[\frac{x}{i}]$, тоест по най-грубата сметка излиза, че от този случай имаме най-много $|N_1| \cdot 2 \sqrt{\sqrt{n}} = 2 n^{\frac{3}{4}}$ числа.

\subsection*{Случай 2: $x \in N_2$}  
Тук работата е малко по пипкава. Пак ще преизползваме старата задача, но ще трябва малко по-внимателно да си направим анализа на сложността. От старото решение можем да заключим, че имаме от този случай най-много $\displaystyle\sum_{x \in N_2} 2 \sqrt{x}$. Това обаче не ни казва много за жалост. Нека го разпишем малко по-подробно. 

\begin{equation*}
    \displaystyle\sum_{x \in N_2} 2 \sqrt{x} = 2\displaystyle\sum_{j=1}^{[\sqrt{n}]} \sqrt{\frac{n}{j}} = 2 \sqrt{n} \displaystyle\sum_{j=1}^{[\sqrt{n}]} \frac{1}{\sqrt{j}} < 2 \sqrt{n} \cdot (2 \sqrt{\sqrt{n}+1} - 1) \approx 4 n^{\frac{3}{4}}
\end{equation*}

Тук използвахме наготово, че $\frac{1}{\sqrt{1}} + \frac{1}{\sqrt{2}} + ... + \frac{1}{\sqrt{m}} \approx 2\sqrt{m+1} - 1$.

\subsection*{Довършване}
\paragraph*{}
Оконачтелно от двата случая имаме, че има горе-долу не повече от $2 n^{\frac{3}{4}} + 4 n^{\frac{3}{4}}$, което е $O(n^{\frac{3}{4}})$. Това трябваше и да се покаже.

\end{document}