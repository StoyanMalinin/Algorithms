\documentclass[12pt]{article}
\usepackage{lingmacros}
\usepackage{tree-dvips}
\usepackage[utf8]{inputenc}
\usepackage[russian]{babel}
\usepackage{amsmath,amssymb}
\usepackage{multirow}
\usepackage{hyperref}
\usepackage{caption}
\usepackage{tabularx}

\begin{document}

\section*{Основни наблюдения}
\paragraph*{}
Тъй като имаме само четири различни вида доминота, можем да се ориентираме към това да правим четиримерно динамично, където всяко измерение е броя на оставащите ни доминота от всеки тип. Това първоначално може да ни изглежда като твърде голяма сложност $120^4$ състояния на пръв поглед, но трябва да се отчете, че бройката реално е много по-малка. Това се дължи на факта, че всъщност не самите бройки на блокчетата са $\leq 120$, а тяхната сума. Това се оказва, че прави много голяма разлика.     

\end{document}