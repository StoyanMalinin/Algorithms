\documentclass[12pt]{article}
\usepackage{lingmacros}
\usepackage{tree-dvips}
\usepackage[utf8]{inputenc}
\usepackage[russian]{babel}
\usepackage{amsmath,amssymb}
\usepackage{multirow}
\usepackage{hyperref}

\usepackage{graphicx}
\graphicspath{ {./images/} }

\begin{document}
\title{Lowest Common Ancestor}
\date{}
\maketitle

\section*{Дефиниция и примери}
\subsection*{Дефиниция}
\paragraph*{}
Нека имаме кореново дърво $T$ с корен $root$. Връх $u$ се нарича предшественик на връх $v$, ако $u$ лежи на единствения път между $root$ и $v$. Ще бележим с $depth(u)$ дълбочината на връх $u$ или с други думи казано - разстоянието от $root$ до $u$. При тази конвенция, най-близкият общ предшественик на $u$ и $v$, който ще бележим с $lca(u, v)$, ще бъде върхът $x$ с максимална дълбочина, който е предшественик едновременно на $u$ и $v$. 
\subsection*{Примери}
\paragraph*{}


\end{document}